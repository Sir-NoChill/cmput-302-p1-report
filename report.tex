\documentclass[sigconf]{acmart}

%% \BibTeX command to typeset BibTeX logo in the docs
\AtBeginDocument{%
	\providecommand\BibTeX{{%
		\normalfont B\kern-0.5em{\scshape i\kern-0.25em b}\kern-0.8em\TeX}}}


\usepackage{lipsum}
%% Rights management information.  This information is sent to you
%% when you complete the rights form.  These commands have SAMPLE
%% values in them; it is your responsibility as an author to replace
%% the commands and values with those provided to you when you
%% complete the rights form.
\copyrightyear{2024}
\acmYear{2024}
\setcopyright{rightsretained}
\acmISBN{N/A}
\acmDOI{N/A}

%% These commands are for a PROCEEDINGS abstract or paper.
\acmConference[CMPUT 302 '24]{}{January 2024}{Edmonton AB}

% shorthand for ualberta definition
\newcommand{\ualberta}{
	\affiliation{%
	  \institution{the University of Alberta}
	  \city{Edmonton}
	  \state{Alberta}
	  \country{Canada}
	}
}

\begin{document}
	%%
	%% The "title" command has an optional parameter,
	%% allowing the author to define a "short title" to be used in page headers.
	\title{CMPUT 302 Deliverable 2}
	
	%% The "author" command and its associated commands are used to define
	%% the authors and their affiliations.
	%% Of note is the shared affiliation of the first two authors, and the
	%% "authornote" and "authornotemark" commands
	%% used to denote shared contribution to the research.
	\author{Ayrton Chilibeck}
	\authornote{All authors contributed equally to this research, and are listed in alphabetical order for simplicity}
	\email{achilibe@ualberta.ca}
	\ualberta

	\author{Eric Kim}
	\authornotemark[1]
	\email{dek@ualberta.ca}
	\ualberta

	\author{Yu Liu}
	\authornotemark[1]
	\email{yliu30@ualberta.ca}
	\ualberta
	
	\author{Vedant Talati}
	\authornotemark[1]
	\email{vtalati@ualberta.ca}
	\ualberta

	\author{Marcus Wilson}
	\authornotemark[1]
	\email{mawilso1@ualberta.ca}
	\ualberta

	%%
	%% By default, the full list of authors will be used in the page
	%% headers. Often, this list is too long, and will overlap
	%% other information printed in the page headers. This command allows
	%% the author to define a more concise list
	%% of authors' names for this purpose.
	\renewcommand{\shortauthors}{Chilibeck, Kim, Liu, Talati, Wilson}
	
	%%
	%% The abstract is a short summary of the work to be presented in the
	%% article.
	\begin{abstract}
	  Report on the severity of and possible remediations for user design errors in Illumia Lab's software %TODO: This is poorly worded
	\end{abstract}

	\keywords{Human-Computer Interaction, UX Design}
	
	\maketitle
	
\section{Introduction}
	\lipsum[2-4]

\section{Errors}
	\lipsum[2-4]

\section{Remediations}
	\lipsum[2-3]

\section{Suggested Roadmap}
	\lipsum[2-3]
	
\section{Conclusion}
	\lipsum[2-2]

\section{Acknowledgments}
	\lipsum[2-2]\cite{Abril07}

	

% bibliography data
\bibliographystyle{ACM-Reference-Format}
\bibliography{bibliography}

\section*{Appendices}
%%
%% If your work has an appendix, this is the place to put it.
\appendix

\section{Research Methods}
	\lipsum[2-2]

\section{Online Resources}
	\lipsum[2-2]
\end{document}
\endinput
%%
%% End of file `sample-lualatex.tex'.
