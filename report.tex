\documentclass[sigconf]{acmart}
% NOTE: To make the documnet anonymous, use the command \documentclass[sigconf, anonymous]{acmart}
%
%% \BibTeX command to typeset BibTeX logo in the docs
\AtBeginDocument{%
	\providecommand\BibTeX{{%
		\normalfont B\kern-0.5em{\scshape i\kern-0.25em b}\kern-0.8em\TeX}}}


\usepackage{lipsum}
%% Rights management information.  This information is sent to you
%% when you complete the rights form.  These commands have SAMPLE
%% values in them; it is your responsibility as an author to replace
%% the commands and values with those provided to you when you
%% complete the rights form.
\copyrightyear{2024}
\acmYear{2024}
\setcopyright{rightsretained}
\acmISBN{N/A}
\acmDOI{N/A}

%% These commands are for a PROCEEDINGS abstract or paper.
\acmConference[CMPUT 302 '24: Illumia Labs]{}{January 2024}{Edmonton AB}

% shorthand for ualberta definition
\newcommand{\ualberta}{
	\affiliation{%
	  \institution{the University of Alberta}
	  \city{Edmonton}
	  \state{Alberta}
	  \country{Canada}
	}
}

\begin{document}
	%%
	%% The "title" command has an optional parameter,
	%% allowing the author to define a "short title" to be used in page headers.
	\title{CMPUT 302 Deliverable 2}
	
	%% The "author" command and its associated commands are used to define
	%% the authors and their affiliations.
	%% Of note is the shared affiliation of the first two authors, and the
	%% "authornote" and "authornotemark" commands
	%% used to denote shared contribution to the research.
	\author{Ayrton Chilibeck}
	\authornote{All authors contributed equally to this research, and are listed in alphabetical order for simplicity}
	\email{achilibe@ualberta.ca}
	\ualberta

	\author{Eric Kim}
	\authornotemark[1]
	\email{dek@ualberta.ca}
	\ualberta

	\author{Yu Liu}
	\authornotemark[1]
	\email{yliu30@ualberta.ca}
	\ualberta
	
	\author{Vedant Talati}
	\authornotemark[1]
	\email{vtalati@ualberta.ca}
	\ualberta

	\author{Marcus Wilson}
	\authornotemark[1]
	\email{mawilso1@ualberta.ca}
	\ualberta

	%%
	%% By default, the full list of authors will be used in the page
	%% headers. Often, this list is too long, and will overlap
	%% other information printed in the page headers. This command allows
	%% the author to define a more concise list
	%% of authors' names for this purpose.
	\renewcommand{\shortauthors}{Chilibeck, Kim, Liu, Talati, Wilson}
	
	%%
	%% The abstract is a short summary of the work to be presented in the
	%% article.
	\begin{abstract}
	  We analyze the functionality and quality of Illumia Lab's \textit{Scenario Builder} to comment on potential improvements and provide a short-term roadmap for development and improvement of the application. We encounter and provide solutions for various problems in the UI, the functionality of the system and the documentation of the program with respect to Human-Computer Interaction principles, Gestalt principles and CRAP design principles. Our solutions follow previously established results from the field of HCI, colour theory as well as results from our experiences as users.
	\end{abstract}

	\keywords{Human-Computer Interaction, UX Design}
	
	\maketitle
	
\section{Introduction}
We evaluated the Illumia Lab's \textit{Scenario Builder} for
\section{System Flaws}
During our exploration and use of the system, we encountered problems in the UI, the program functionality and the documentation of the program. We outline the most important findings in the following sections.
\subsection{UI}
Our results from UI analysis are largely cosmetic, but the current state of the software impedes effective use of the system by the end users. The layout of the system does not efficiently show the information in a given scene and the process of changing a scene takes a large amount of work from the user. Additionally, the system does not have clear indication of the correct user actions and fails to introduce the user to the potential actions at any given point in the scene building process.

\subsubsection{Colour-Scheme}
The current colour scheme (Purple (\verb|#07012F|), Blue (\verb|#0191FD|) and Red (\verb|#FC5C00|)) is jarring to the eyes. The literature establishes that particularly blue and purple are hard for people to look at for extended periods of time \cite{HCI-Dev}.

\subsubsection{Tab Display}
The current display of tabs in the scene builder fails to effectively show the user the state of the program. Tabs for each scene do not give the user context on the scene's purpose or the information contained therein. The preview pane attempts to mitigate these shortcomings, but the scene-graph display is lacking in relationships to other scenes.

\subsubsection{Preview Pane}
The alignment in the preview pane is poor, in addition to an absence of dynamic sizing of the screen (for mobile and re-sizable web pages) the utility of the data presented is questionable.

\subsubsection{Ease of Use}
Building a scene currently takes a minimum of 9 clicks. Although the community has debunked the '3-click rule' \cite{WP-3CR}, the importance of ease of access for information is still paramount in design. Current research into the concept of 'Interaction Elasticity' \cite{IE-NG} rather enforces the significance of eliminating useless interaction. Currently, the scene builder presents the user with a great deal of useless interaction in the form of these clicks.

\subsection{Functionality}
\subsection{Documentation}

\section{Remediations}
	\lipsum[2-3]
\subsection{UI}
\subsection{Functionality}
\subsection{Documentation}

\section{Suggested Roadmap}
	\lipsum[2-3]
\subsection{Scene Representations}
\subsection{Documentation}
\subsection{Effective UI}
\subsection{Color Scheme}
	
\section{Conclusion}
	\lipsum[2-2]

\section{Acknowledgments}
	\lipsum[2-2]\cite{Abril07}

% bibliography data
\bibliographystyle{ACM-Reference-Format}
\bibliography{bibliography}

\section*{Appendices}
%%
%% If your work has an appendix, this is the place to put it.
\appendix

\section{Research Methods}
	\lipsum[2-2]

\section{Online Resources}
	\lipsum[2-2]
\end{document}
\endinput
%%
%% End of file `sample-lualatex.tex'.
